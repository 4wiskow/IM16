\documentclass[a4paper,10pt]{article}

% Hier die Nummer des Blatts und Autoren angeben.
\newcommand{\blatt}{9}
\newcommand{\autor}{Georg Scherer}

\usepackage{hci}
\usepackage{textcomp}
\usepackage{eurosym}

\begin{document}
% Seitenkopf mit Informationen
\kopf
\renewcommand{\figurename}{Figure}

\aufgabe{1}
\begin{enumerate}
\item
Up-Verfahren für Vorfilterung, da die Farbwerte in aufsteigender Folge von oben links nach unten rechts vergrößern und so hohe Werte gefiltert werden, hauptsächlich 0 auftaucht \\
\begin{tabular}{|c|c|c|c|c|c|c|c|}
\hline
1&1&1&1&2&3&3&3 \\
\hline
1&1&1&1&1&1&1&1 \\
\hline
1&1&1&1&0&0&0&0 \\
\hline
0&0&0&0&0&0&0&1 \\
\hline
0&0&0&0&0&1&1&0 \\
\hline
1&1&1&1&1&1&1&1 \\
\hline
1&1&1&1&1&1&1&1 \\
\hline
1&1&1&1&1&1&1&1 \\
\hline
\end{tabular}
\item Die Werte 0, 1, 2 und 3 sind vorhanden, demnach besitzt das Anfangswörterbuch diese 4 Einträge.
\item Auftrittshäufigkeiten:
2: 1, 3: 3, 0: 17, 1: 43 \\
Damit ergibt sich erst das Paar 2\&3, dann (2,3)\&0, dann alle. Kodierung damit: \\
1 : 1,
0 : 01,
2 : 001,
3 : 000
\item
1111001000000000 11111111 111101010101 010101010101011 01010101011101 11111111 11111111 11111111 
\end{enumerate}
\aufgabe{2}
\begin{figure}[hb]
\centering
\includegraphics[width=0.4\textwidth]{RGBCube}
\caption{RGB-Farbwürfel}
\end{figure}
CMY umsschließt im Farbwürfel das gegenüberliegende Dreieck des RGB-Systems: beide lassen sich prolemlos ineinander umrechen. 
Für die Darstellung von Bildern würde dies bedeuten, dass durch die additive Farbmischung bei CMY jeder Pixel dauerhaft angestrahlt werden müsste, wobei dann pro Pixel einzelne Farben herausgefiltert werden.
\aufgabe{3}
Die Komplementärfarben sind eine Spiegelung der Ursprungsfarbe: Die Komplementärfarbe eines RGB-Wertes ist demnach dessen CMY-Darstellung im RGB-System. Jeder Wert wird somit gespiegelt. Selbes gilt dadurch selbstverständlich auch, wenn der Ursprung CMY ist.\\
$(R,G,B) = (0.8,0.1,1.0)$   $Komplementaer = (C,M,Y) = (0.2,0.9,0.0)$
\aufgabe{4}
\begin{enumerate}
\item Text Cyan, Papier Gelb
\item Text Schwarz, Papier Cyan
\item Text Rot, Papier Magenta
\item Text Schwarz, Papier Cyan
\item Text Magenta, Papier Gelb
\item Text orange, Papier helles Gelb
\end{enumerate}
\end{document}
