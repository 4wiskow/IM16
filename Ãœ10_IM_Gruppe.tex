\documentclass[a4paper,10pt]{article}

% Hier die Nummer des Blatts und Autoren angeben.
\newcommand{\blatt}{9}
\newcommand{\autor}{Georg Scherer}

\usepackage{hci}
\usepackage{textcomp}
\usepackage{eurosym}

\begin{document}
% Seitenkopf mit Informationen
\kopf
\renewcommand{\figurename}{Figure}

\aufgabe{1}
\begin{enumerate}
\item
Up-Verfahren für Vorfilterung, da die Farbwerte in aufsteigender Folge von oben links nach unten rechts vergrößern und so hohe Werte gefiltert werden, hauptsächlich 0 auftaucht \\
\begin{tabular}{|c|c|c|c|c|c|c|c|}
\hline
1&1&1&1&2&3&3&3 \\
\hline
1&1&1&1&1&1&1&1 \\
\hline
1&1&1&1&0&0&0&0 \\
\hline
0&0&0&0&0&0&0&1 \\
\hline
0&0&0&0&0&1&1&0 \\
\hline
1&1&1&1&1&1&1&1 \\
\hline
1&1&1&1&1&1&1&1 \\
\hline
1&1&1&1&1&1&1&1 \\
\hline
\end{tabular}
\item Die Werte 0, 1, 2 und 3 sind vorhanden, demnach besitzt das Anfangswörterbuch diese 4 Einträge.
\item Auftrittshäufigkeiten:
2: 1, 3: 3, 0: 17, 1: 43 \\
Damit ergibt sich erst das Paar 2\&3, dann (2,3)\&0, dann alle. Kodierung damit: \\
1 : 1,
0 : 01,
2 : 001,
3 : 000
\item
1111001000000000 11111111 111101010101 010101010101011 01010101011101 11111111 11111111 11111111 
\end{enumerate}
\end{document}
